\documentclass[xcolor=table]{beamer}

\usepackage[utf8]{inputenc}
\usepackage[T1]{fontenc}
\usepackage[frenchb]{babel}
\usepackage{amsmath}
\usepackage{xcolor}
\usepackage{graphicx}
\usepackage{multirow}
\usepackage{array}
\usepackage{verbatim}

\usetheme{Berlin}
%\usecolortheme{beaver}

\setbeamertemplate{navigation symbols}{}

\setbeamercolor{note}{fg=black,bg=lightgray} 

\title{Classification non supervisée de photos}
\subtitle{Fouille d'images}
\author{Aurélien Cavelan, Jérôme Richard}
\institute{Université d'Orléans}
\date{Mars 2014}

\begin{document}

% === PRESENTATION =====================================================
	
\frame{\titlepage}

% ======================================================================

\AtBeginSection[]{
	\begin{frame}<beamer>
	\frametitle{Plan}
	\tableofcontents[currentsection,currentsubsection]
	\end{frame}
}
	
\AtBeginSubsection[]{
	\begin{frame}<beamer>
	\frametitle{Plan}
	\tableofcontents[currentsection,currentsubsection]
	\end{frame}
}
	
% === DIAPOS ===========================================================

\section{Choix logiciels}

	\begin{frame}{Approche}
		\begin{block}{Bibliothèques et langages}
			\begin{itemize}
		        \item Matlab
		        \item C++ - OpenCV
		        \item Python - Scikit / Skimage
		        \item Python - OpenCV
	    	\end{itemize}
    	\end{block}
	\end{frame}

	\begin{frame}{Structure du projet}
		\begin{block}{Architecture}
			\begin{itemize}
				\item Code C++ / OpenCV
				\item Méthode d'apprentissage par validation croisée
				\item Séparation des descripteurs et du code principal
				\item Possibilité d'ajouter simplement des descripteurs
			\end{itemize}
		\end{block}
	\end{frame}

	\begin{frame}{Apprentissage non-supervisé}
		\begin{block}{Validation croisée}
			\begin{itemize}
				\item Tirage de plusieurs images aléatoirement
				\item Extraction des descripteurs d'images
				\item Classification (kmeans ou hiérarchique)
				\item Association d'une classe à de nouvelles images
				\item Calcul de l'index de Rand
			\end{itemize}
		\end{block}
	\end{frame}

% === DESCRIPTEURS =====================================================

\section{Descripteurs}

	\begin{frame}{Descripteurs 1D}
		\begin{block}{Descripteurs simples}
			\begin{itemize}
				\item Moyenne
				\item Variance
				\item Taille de l'image
			\end{itemize}
		\end{block}
		\begin{block}{Résultats}
			Résultats médiocres (Rand : 0.5 - 0.65), peu résistants aux variations de lumière et indiférrents aux formes.
		\end{block}
	\end{frame}

	\begin{frame}{Moments de Hu}
		\begin{block}{Avantages}
			\begin{itemize}
				\item Descripteur de formes
				\item Détection de plusieurs objets
				\item Suppose une bonne détection de contours
			\end{itemize}
		\end{block}
		\begin{block}{Inconvéniants}
			\begin{itemize}
				\item Non adapté à des images complexes
				\item Génère de nombreux points extrêmes
				\item La densité des données est hétérogène
			\end{itemize}
		\end{block}
	\end{frame}

	\begin{frame}{SIFT / SURF}
		\begin{block}{Méthode}
			\begin{enumerate}
				\item Extraction de points clés sur plusieurs images
				\item Création d'un dictionnaire de mots visuels (kmeans)
				\item Extraction de descripteurs en utilisant le dictionnaire
				\item Cela revient à comparer des histogrammes de distribution
			\end{enumerate}
		\end{block}
		\begin{block}{Résultats}
			Enregistrement du dictionnaire et des descripteurs dans des fichiers.\\
			Résultats assez bon (Rand : 0.7 - 0.8). Les résultats varies en fonction de la taille du dictionnaire.
		\end{block}
	\end{frame}

% === SYNTHESE =====================================================

\section{Résultats}

	\begin{frame}{Tableaux}
		\begin{figure}[!h]
			\centering
			\begin{tabular}{ | l | c | l | }
				\hline
				Descripteurs & Pré-traitement & Rand \\
				\hline
					Moyenne &	-			 & 0.65\\
					Gabor	&	-			 & 0.67\\
					SURF 	& 	blur + canny & 0.66\\
					SIFT 	&	blur + canny & 0.68\\
					SURF 	& 	- 			 & 0.72\\
					SIFT 	&	-			 & 0.78\\
				\hline  
			\end{tabular}
		\end{figure}
	\end{frame}

	\section{Questions}
	\begin{frame}{Questions}
	    \begin{beamercolorbox}[center,shadow=true,rounded=true,]{note} 
	        \huge{Questions?}
	    \end{beamercolorbox} 
	\end{frame} 

\end{document}
