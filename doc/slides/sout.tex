\documentclass[xcolor=table]{beamer}

\usepackage[utf8]{inputenc}
\usepackage[T1]{fontenc}
\usepackage[frenchb]{babel}
\usepackage{amsmath}
\usepackage{xcolor}
\usepackage{graphicx}
\usepackage{multirow}
\usepackage{array}
\usepackage{verbatim}

\usetheme{Berlin}
%\usecolortheme{beaver}

\setbeamertemplate{navigation symbols}{}

\setbeamercolor{note}{fg=black,bg=lightgray} 

\title{Classification non supervisée de photos}
\subtitle{Fouille d'images}
\author{Aurélien Cavelan, Jérôme Richard}
\institute{Université d'Orléans}
\date{Mars 2014}

\begin{document}

% === PRESENTATION =====================================================
	
\frame{\titlepage}

% ======================================================================

\AtBeginSection[]{
	\begin{frame}<beamer>
	\frametitle{Plan}
	\tableofcontents[currentsection,currentsubsection]
	\end{frame}
}
	
\AtBeginSubsection[]{
	\begin{frame}<beamer>
	\frametitle{Plan}
	\tableofcontents[currentsection,currentsubsection]
	\end{frame}
}
	
% === DIAPOS ===========================================================

\section{Choix logiciels}

	\begin{frame}{Approche}
		\begin{itemize}
			\item En fonction du voisinage
			\item Sans a priori sur le réseau global
			\item Avec une restriction sur la taille des messages
		\end{itemize}
	\end{frame}

	\begin{frame}{Apprentissage}
		\begin{itemize}
			\item En fonction du voisinage
			\item Sans a priori sur le réseau global
			\item Avec une restriction sur la taille des messages
		\end{itemize}
	\end{frame}

	\begin{frame}{Architecture}
		\begin{itemize}
			\item En fonction du voisinage
			\item Sans a priori sur le réseau global
			\item Avec une restriction sur la taille des messages
		\end{itemize}
	\end{frame}

% === DESCRIPTEURS =====================================================

\section{Descripteurs}

	\begin{frame}{Descripteurs 1D}
		\begin{itemize}
			\item En fonction du voisinage
			\item Sans a priori sur le réseau global
			\item Avec une restriction sur la taille des messages
		\end{itemize}
	\end{frame}

	\begin{frame}{Moments de Hu}
		\begin{itemize}
			\item En fonction du voisinage
			\item Sans a priori sur le réseau global
			\item Avec une restriction sur la taille des messages
		\end{itemize}
	\end{frame}

	\begin{frame}{SIFT / SIRF}
		\begin{itemize}
			\item En fonction du voisinage
			\item Sans a priori sur le réseau global
			\item Avec une restriction sur la taille des messages
		\end{itemize}
	\end{frame}

% === SYNTHESE =====================================================
\section{Synthèse}

	\begin{frame}{Tableaux}
		\begin{itemize}
			\item En fonction du voisinage
			\item Sans a priori sur le réseau global
			\item Avec une restriction sur la taille des messages
		\end{itemize}
	\end{frame}

	\begin{frame}{Conclusion}
		\begin{itemize}
			\item En fonction du voisinage
			\item Sans a priori sur le réseau global
			\item Avec une restriction sur la taille des messages
		\end{itemize}
	\end{frame}

\end{document}
